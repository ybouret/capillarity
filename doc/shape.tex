\documentclass[aps,onecolumn]{revtex4}
\usepackage{graphicx}
\usepackage{amssymb,amsfonts,amsmath,amsthm}
\usepackage{chemarr}
\usepackage{bm}
\usepackage{pslatex}
\usepackage{mathptmx}
\usepackage{xfrac}
\usepackage{xcolor}

\begin{document}

\section{Curvature of a revolution surface}

If we have
\begin{equation}
	r(t),\;\; z(t)
\end{equation}
then the mean curvature is equal to
\begin{equation}
	\mathcal{H} =  \dfrac{
		r \left(r''z'-r'z''\right)  - z' \left(r'^2+z'^2\right)
	}
	{
		2r\left(r'^2+z'^2\right)^{3/2}
	}
\end{equation}
We know use an intrinsic representation of the $(r,z)$ profile using the
angle $\phi$ and the variable $s$ so that
\begin{equation}
\left\lbrace
	\begin{array}{rcl}
	r' & = & \cos \phi\\
	z' & = & \sin \phi\\
	r'' & = & -\sin\phi \phi'\\
	z'' & = & \cos\phi  \phi'\\
	\end{array}
\right.
\end{equation}
and we get
\begin{equation}
	\mathcal{H} = -\frac{1}{2} \left(\phi'+\dfrac{\sin\phi}{r}\right)
\end{equation}

\section{Surface equation}
Since the capillary pressure is supporting the liquid
\begin{equation}
	\gamma \mathcal{H} = \rho g z
\end{equation}
or using the capillary length
\begin{equation}
	\lambda = \sqrt{\dfrac{\gamma}{\rho g}}
\end{equation}

\begin{equation}
	\mathcal{H} = \dfrac{z}{\lambda^2}
\end{equation}

\section{Bridge description}
\subsection{Polar Equation}
Let $\rho(\alpha)$ be a polar description of the lens, so that the bridge description in the $xy$ plane are
\begin{equation}
	\left\lbrace
	\begin{array}{rcl}
	x(\alpha) & = & \rho(\alpha)\sin\alpha\\
	y(\alpha) & = & h + R - \rho(\alpha)\cos\alpha\\
	\rho(0)   & = & R\\
	\rho'(0)  & = & 0\\
	\end{array}
	\right.
\end{equation}

\subsection{Contact Angle and Initial Profile Angle}
The   tangent vector is
\begin{equation}
	\vec{\tau}_\alpha = 
	\dfrac{1}{\sqrt{\rho^2+\rho'^2}}
	\begin{pmatrix}
		\rho'\sin\alpha+\rho\cos\alpha\\
		\rho\sin\alpha - \rho'\cos\alpha\\
	\end{pmatrix}
 = 
 	\begin{pmatrix}
	\cos\omega\\
	\sin\omega
	\end{pmatrix}
\end{equation}
so that
\begin{equation}
	\omega = \alpha - \arcsin\left(\dfrac{\rho'}{\sqrt{\rho^2+\rho'^2}}\right)
\end{equation}
Remark: the curvature is
\begin{equation}
	\gamma_\alpha = \dfrac{\rho^2 + 2\rho'^2 - \rho\rho''}{\left(\rho^2+\rho'^2\right)^{3/2}}
\end{equation}
So that the initial angle is
\begin{equation}
	\phi_0 = \omega - \theta \;\; \text{\textcolor{red}{CHECK!!!}}
\end{equation}


\subsection{Spherical Matching}
Let us assume that we observe only a part of the bridge, but we numerically need
a description for $\alpha\in[0,\pi]$.\\

Then we assume that we have fit of $\rho(\alpha)$ for $\alpha\in[0,\beta]$.
We need a $\mathcal{C}^1$ continuation of $\rho$ for $\alpha\in[\beta,\pi]$, so
that the tangent angle is continuous
The continuous fit gives us $\rho(\beta),\rho'(\beta)$
We need to expand it to $\rho_\pi,\rho'(\pi)=0$.
Using an even description,
\begin{equation}
	\tilde{\rho}(\alpha) = \rho_\pi + P\left( \left(\dfrac{\pi-\alpha}{\pi-\beta}\right)^2 \right)
	, \;\; P(X) = \dfrac{UX+VX^2}{2}
\end{equation}
with
\begin{equation}
\left\lbrace
	\begin{array}{rcl}
	U & = & -4\left(\rho_\pi-\rho_\beta\right)+\left(\pi-\beta\right) \rho'_\beta\\
	V & = & 2\left(\rho_\pi-\rho_\beta\right)-\left(\pi-\beta\right) \rho'_\beta \\
	\end{array}
\right.
\end{equation}

We may set 
\begin{equation}
	\rho_\pi \equiv \dfrac{1}{\beta}\int_0^\beta R(\alpha)\,\mathrm{d}\alpha
\end{equation}

\subsection{Finding the Lens Representation}
We assume that we get a set of points representing the lens, namely $\lbrace x_i,y_i \rbrace_{i\in[1:N]}$ that must
fit.

\begin{equation}
	\left\lbrace
		\begin{array}{rcl}
	x(\alpha_i) & = & X_c + \rho_i\sin\alpha_i\\
	y(\alpha_i) & = & Y_c - \rho_i\cos\alpha_i\\
		\end{array}
	\right.
\end{equation}
If we choose $X_c,Y_c$, then
\begin{equation}
	\left\lbrace
	\begin{array}{rcl}	
	\rho_i   &=&\sqrt{\left(x_i-X_c\right)^2+\left(Y_c-y_i\right)^2}\\
	\alpha_i &=& 2 \arctan\left[ \dfrac{\left(x_i-X_c\right)}{\left(Y_c-y_i\right)+\rho_i}\right]\\
	\end{array}
	\right.
\end{equation}
The $X_c,Y_c$ values are determined by the minimization of $\sigma^2_\rho$.

\end{document}

\section{Curvature of a revolution surface}

If we have
\begin{equation}
	r = \phi(t),\;\; z = \psi(t)
\end{equation}
then the mean curvature is equal to
\begin{equation}
	\mathcal{H} =  
	\dfrac{\phi\left(\phi''\psi'-\phi'\psi''\right) - \psi'\left({\phi'}^2+{\psi'}^2\right)}
	{2\vert\phi\vert\left({\phi'}^2+{\psi'}^2\right)^{3/2}}
\end{equation}
or
\begin{equation}
	\mathcal{H} =  \dfrac{
		r \left(r''z'-r'z''\right)  - z' \left(r'^2+z'^2\right)
	}
	{
		2r\left(r'^2+z'^2\right)^{3/2}
	}
\end{equation}

Using a $z$-major description, namely $z(t)=t$, we get $\Psi'=1$ and $\Psi''=0$
\begin{equation}
	\mathcal{H} =  \dfrac{r r'' - \left(1+{r'}^2\right)}{2r\left(1+{r'}^2\right)^{3/2}}
\end{equation}

\section{Surface equation}
Since the capillary pressure is supporting the liquid
\begin{equation}
	\gamma \mathcal{H} = \rho g z
\end{equation}
and finally
\begin{equation}
	\dfrac{r''}{\left(1+{r'}^2\right)^{3/2}} - \dfrac{1}{r\left(1+{r'}^2\right)^{1/2}} = \kappa^2 z 
\end{equation}

The contact point $(r_c,z_c)$ is picked up by the angle $\alpha$ so that
\begin{equation}
	r_c = R\sin\alpha,\;z_c = h+R\left(1-\cos\alpha\right)
\end{equation}
If the contact angle is $\theta$, then
\begin{equation}
	\dot{r}_c = \cot\left(\alpha+\theta\right).
\end{equation}

What is $\alpha$ such that
\begin{equation}
	\lim_{z\to0}\dot{r} = \infty
\end{equation}

\section{Normalisation}
We use
\begin{equation}
	r = R\cdot u,\; z = R \cdot y
\end{equation}
to get
\begin{equation}
	\dfrac{u''}{\left(1+{u'}^2\right)^{\frac{3}{2}}} - \dfrac{1}{u\left(1+{u'}^2\right)^{\frac{1}{2}}} = \underbrace{\left(R\kappa\right)^2}_{K^2} y,
\end{equation}
to be solved with
\begin{equation}
	u_c = \sin\alpha, \; u'_c = \cot\left(\alpha+\theta\right), \; y_c = \underbrace{\dfrac{h}{R}}_{\beta} + \left(1-\cos\alpha\right)
\end{equation}

We notice that
\begin{equation}
	- \partial_y \left(\dfrac{u}{\sqrt{1+{u'}^2}}\right) = u u' \left[ \dfrac{u''}{\left(1+{u'}^2\right)^{\frac{3}{2}}} - \dfrac{1}{u\left(1+{u'}^2\right)^{\frac{1}{2}}} \right]
\end{equation}


\section{Using other parametric stuff}
We use
\begin{equation}
	\mathcal{H} =  \dfrac{
		r \left(r''z'-r'z''\right)  - z' \left(r'^2+z'^2\right)
	}
	{
		r\left(r'^2+z'^2\right)^{3/2}
	} = \kappa^2 z
\end{equation}
We now use the intrinsic angular description 
\begin{equation}
	r' = \cos \phi,\; z' = \sin \phi
\end{equation}
so that
\begin{equation}
	r'' = -\sin \phi\, \phi',\; z'' = \cos \phi \, \phi'
\end{equation}



\section{Parametric Description}
\subsection{New Equations}
Let $\rho(\alpha)$ be a polar description of the lens, so that the lens coordinates in the $xy$ plane are
\begin{equation}
	\left\lbrace
	\begin{array}{rcl}
	x(\alpha) & = & \rho(\alpha)\sin\alpha\\
	y(\alpha) & = & h + R - \rho(\alpha)\cos\alpha\\
	\end{array}
	\right.
\end{equation}
The   tangent vector is
\begin{equation}
	\vec{\tau}_\alpha = 
	\dfrac{1}{\sqrt{\rho^2+\rho'^2}}
	\begin{pmatrix}
		\rho'\sin\alpha+\rho\cos\alpha\\
		\rho\sin\alpha - \rho'\cos\alpha\\
	\end{pmatrix}
 = 
 	\begin{pmatrix}
	\cos\omega\\
	\sin\omega
	\end{pmatrix}
\end{equation}
so that
\begin{equation}
	\omega = \alpha - \arcsin\left(\dfrac{\rho'}{\sqrt{\rho^2+\rho'^2}}\right)
\end{equation}
Remark: the curvature is
\begin{equation}
	\gamma_\alpha = \dfrac{\rho^2 + 2\rho'^2 - \rho\rho''}{\left(\rho^2+\rho'^2\right)^{3/2}}
\end{equation}

The new contact conditions are
\begin{equation}
	\left\lbrace
		\begin{array}{rcl}
		r_c & = & \rho(\alpha)\sin(\alpha)\\
		y_c & = & h + R - \rho(\alpha)\cos(\alpha)\\
		r_c' & = & \cot\left(\omega+\theta\right)\\
		\end{array}
	\right.
\end{equation}

\subsection{Finding the Lens Representation}
We assume that we get a set of points representing the lens, namely $\lbrace x_i,y_i \rbrace_{i\in[1:N]}$ that must
fit
\begin{equation}
	\left\lbrace
		\begin{array}{rcl}
	x(\alpha_i) & = & X_c + \rho_i\sin\alpha_i\\
	y(\alpha_i) & = & Y_c - \rho_i\cos\alpha_i\\
		\end{array}
	\right.
\end{equation}
If we choose $X_c,Y_c$, then
\begin{equation}
	\left\lbrace
	\begin{array}{rcl}	
	\rho_i   &=&\sqrt{\left(x_i-X_c\right)^2+\left(Y_c-y_i\right)^2}\\
	\alpha_i &=& 2 \arctan\left[ \dfrac{\left(x_i-X_c\right)}{\left(Y_c-y_i\right)+\rho_i}\right]\\
	\end{array}
	\right.
\end{equation}
The $X_c,Y_c$ values are determined by the minimization of $\sigma^2_\rho$.




\end{document}